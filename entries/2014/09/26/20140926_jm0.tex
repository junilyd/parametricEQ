% mainfile: ../../../../master.tex
\subsection{Old Quiz From 2013}
\label{sec:Old Exercises from 2014}
    These exercises has been answered during summer holiday. The first part is a quiz.
    \begin{itemize}
        \item What is a linear system, and what is meant by its solution?
    \end{itemize}
    A system of linear equations is a collection of one or more linear equations involving the same variables. 
    which is not squared or "similar multiplied" (lines or hyper planes).
    The solution set is all possible solutions which maps the system 'A' to the desired vector 'b'. (x=A\b)
    solution if b is in the column space of A (b is linear combination of the columns of A).
\begin{itemize}
    \item What do we mean by consistency of a linear system?
\end{itemize}
    There is either none, one or infinitely many solutions. 
    A system of linear equations is said to be consistent if it has either one solution or infinitely many solutions; 
    a system is inconsistent if it has no solution.
\begin{itemize}
    \item Explain what is echelon form and reduced echelon form.
\end{itemize}
    Both are triangular form.
    Echelon Form is when the system is brought to upper triangular form, where all leading entries is the only non-zero in its column. (pivot in all rows except zero-rows.)
    Reduced Echelon Form is when every leading entry is 1 with zeros above and below.
\begin{itemize}
    \item Explain the meaning of the following terms in relation to echelon forms: pivot, pivot positions, pivot columns.
\end{itemize}
    pivot: leading entry.
    pivot position: A pivot position is a location that corresponds to a leading entry in the reduced echelon form of a matrix.
    pivot column: A pivot column contains a pivot position. 
\begin{itemize}
    \item What is a linear combination of vectors, and what is the relation to Span{. . . }?
\end{itemize}    
    y = c1v1 + ··· + cpvp is a linear combination, where c is scalars and v is vectors.
    the span{v1, v2, v3} is a subset of R3 spanned by v1, v2 and v3; That is all linear combinations of v1, v2 qnd v3
    . see example 6 pp. 36.
\begin{itemize}
    \item How can we write a linear system as a matrix equation?
\end{itemize}
    see pp. 33.
\begin{itemize}
    \item Define independence/dependence in relation to vectors. 
\end{itemize}
    if two vectors can be expressed as linear combinations of each other they are linear dependent and vice versa.
    in a matrix system Ax=0 has only the trivial solution if it is linear independent set.
    if a set contains more vectors that entries in each vector, the set is linear dependent.
\begin{itemize}
    \item Explain how we sum, multiply and transpose matrices.
\end{itemize}
    sum : each corresponding element
    mult: inner product of row and column.
    tran: inner product of row to col and complex conjugate.
\begin{itemize}
    \item What do we mean by a matrix inverse?
    \item When can we invert a matrix? Mention some of the criteria.
\end{itemize}

\subsection{Exercise 1 as MATLAB listing}
\label{sec:Exercise 1 as MATLAB listing}
\lstinputlisting{m-filer/mspr_old1.m}

\subsection{Quiz 2}
\label{sec:Quiz_1}

\subsubsection{Quiz} 1: What do we mean by a transformation T from Rn to Rm, and what is the meaning of its domain, codomain and range?i
        A transformation (or function or mapping) T from Rn to Rm is a rule that assigns to each vector x in Rn 
        a vector T(x) in Rm. 
        The set Rn is called the domain of T.
        Rm is called the codomain of T.
        For x in Rn, the vector T(x) in Rm is called the image of x (under the action of T ). 
        The set of all images T (x) is called the range of T .

\subsubsection{Quiz} 2: What is a linear transformation, and how is it related to a matrix transformation?
        Every matrix transformation is a linear transformation.

\subsubsection{Quiz} 3: Regarding mappings, what do we mean by the terms "onto" and "one- to-one"?
        A mapping$ T : Rn \rightarrow Rm $ is said to be onto Rm if each b in Rm is the image of at least one x in Rn. 
        A mapping$ T : Rn \rightarrow Rm $ is said to be one-to-one if each b in Rm is the image of at most one x in Rn.
        (pp. 87)

\subsubsection{Quiz} 4: Explain the principle behind the LU factorization.
        Lower and upper trinagular - L:lower - U upper triangular row echelon form of the given matrix.

\subsubsection{Quiz} 5: What does |A| = 0 imply?
        A is not invertible.

\subsubsection{Quiz} 6: How can we find the matrix inverse, using determinants?
        $A^-1 = 1/det(A)*adj(A)$, where adj(A) is the adjugate of A. (the matrix of co-factors (pp. 204))

\subsubsection{Quiz} 7: What is the relation between areas/volumes of parallelograms and parallelepipeds and determinants?
        If A is a 2×2 matrix, the area of the parallelogram determined by the columns of A is |det A|. 
        If A is a 3×3 matrix, the volume of the parallelepiped determined by the columns of A is |det A|.

\subsubsection{Quiz} 8: Answer Q7, for linear transformations of parallelograms and parallelepipeds.

\subsection{Exercise 2 as MATLAB listing}
\label{sec:Exercise 2 as MATLAB listing}
\lstinputlisting{m-filer/mspr_old2.m}

\subsection{Quiz 3}
\label{sec:Quiz_3}

\subsubsection{Quiz} 1: What is eigenvalues and -vectors?
        An eigenvector of an n×n matrix A is a nonzero vector x such that $Ax = \lambda x$ for some scalar $\lambda$. 
        A scalar $\lambda$ is called an eigenvalue of A if there is a nontrivial solution x of $Ax = \lambda x$; 
        such an x is called an eigenvector corresponding to $\lambda$. (pp.303)

\subsubsection{Quiz} 2: How can we find the eigenvalues of a triangular matrix?
        The eigenvalues of a triangular matrix are the entries on its main diagonal.

\subsubsection{Quiz} 3: What is the characteristic equation?
        Useful information about the eigenvalues of a square matrix A is encoded in $A-\lambda I$. 
        This matrix fails to be invertible precisely when its determinant is zero. 
        So the eigenvalues of A are the solutions of the equation $det(A-\lambda I)=0$, 
        which is the characteristic equation.

\subsubsection{Quiz} 4: What does it mean that a matrix is diagonalizable, and how can such a matrix be diagonalized?
        A square matrix A is said to be diagonalizable if A is similar to a diagonal matrix, that is, 
        if $A = PDP^{-1}$ for some invertible matrix P and some diagonal matrix D.
        $A^3 = P  D^3 * P^{-1}$. D has the eigenvalues on the diagonal. P contains the eigenspace which spans Rn.


\subsubsection{Quiz} 5: Mention a condition for a matrix to be diagonalizable?
        An n×n matrix A is diagonalizable if and only if A has n linearly independent eigenvectors.
        In fact,$ A = PDP -1$ , with $D$ a diagonal matrix, if and only if the columns of P are n linearly independent
        eigenvectors of A. In this case, the diagonal entries of D are eigenvalues of A that correspond,
        respectively, to the eigenvectors in P.
        In other words, A is diagonalizable if and only if there are enough eigenvectors to
        form a basis of Rn. We call such a basis an eigenvector basis.

\subsubsection{Quiz} 6: What are the princples behind the (inverse) power methods?
        $A^k = P * D^k * P^-1.$

\subsubsection{Quiz} 7: Explain the following terms in relation to vectors: inner product, length, orthogonal, distance, projection.

\subsubsection{Quiz} 8: What is the characteristics of orthogonal and orthonormal vector sets?

\subsubsection{Quiz} 9: What is the general least squares problem? Mention some real-life examples of least squares problems.

% The part of the label after the colon must match the file name. Otherwise,
% conditional compilation based on task labels does NOT work.
\label{task:20140926_jm0}
\tags{courses,mspr}
\authors{jm}
%\files{}
%\persons{}
