% mainfile: ../../../../master.tex
The project is planned to be carried out as a mini-project.
A mini project is required as an evaluation subject for discussion at the examination.
A parametric equalizer is normally the first step after the gain control in an audio-mixer channel.
Therefore, it is of great interest to design such a technology.
This is being planned to design a digital equalizer and it would be awesome to implement it as a VST-plugin in C++.
\subsection{Starting the Equalizer Project}
The starting point is building the EQ from the book by Orfanidis called Introduction to Signal Processing(reference will probably not be put in this worksheet).
The basic function on $z$-domain is described as
\begin{equation}
    H(z) = \frac{ \frac{G0-GB}{1+\beta} -2 \frac{G0\cos{\omega_0} }{1+\beta} z^{-1} + \frac{G0-GB}{1+\beta}z^{-2}} { 1 - 2\frac{\cos{\omega_0}}{1+\beta}z^{-1} + \frac{1-\beta}{1+\beta}z^{-2}  }
\end{equation}
where
\begin{itemize}
\item $\beta = \sqrt{\frac{GB^2-G0^2}{G^2-GB^2}\tan{\frac{\Delta\omega}{2}}}$
\item $GB$ is the gain at the cut off frequency.
\item $G0$ is the reference gain, which usually is prefered to be 1 for unity gain.
\item $G$ is the gain factor, which can be used for both boost and cut.
\item The tangent part is derived through the bilinear transform.
\end{itemize}
% The part of the label after the colon must match the file name. Otherwise,
% conditional compilation based on task labels does NOT work.
\label{task:20140925_jm0}
\tags{eq,start}
\authors{jm}
%\files{}
%\persons{}
